\begin{abstract}

الگوریتم‌های فشرده‌سازی در طی عمر علوم کامپیوتر یکی از مهم‌ترین موضوعات مورد بحث آن بوده اند. در ابتدا با توجه
به کندی و ضعف سخت‌افزار و نیاز به انتقال حداقلی داده این الگوریتم‌ها مورد توجه قرار گرفتند؛ سپس با پیشرفت اینترنت و نیاز به
انتقال سریع داده، این الگوریتم‌ها دوباره به یکی از مهم‌ترین زمینه‌های تحقیق در علوم کامپیوتر تبدیل شدند، در این
زمان بسیاری از الگوریتم‌های مهم فشرده‌سازی داده کشف شدند. فرمت‌هایی هم‌چون \lr{mp3} یا 
jpeg 
- که از شناخته‌ترین فرمت‌های فشرده‌سازی حال حاضر هستند - ماحصل تحقیقات همین دوران اند. الگوریتم‌های فشرده‌سازی نه تنها راه خود را به دنیای مدیا
باز کردند بلکه در پردازش سیگنال نیز تاثیر عمده‌ای داشتند، تمامی ارتباطات رادیویی با استفاده از این الگوریتم‌ها سعی در 
بالا بردن امنیت و کاهش هزینه انتقال داده دارند، در رمزنگاری نیز این الگوریتم‌ها راه‌گشایی می‌کنند و بسیاری از الگوریتم‌های رمز‌نگاری
و درهم‌سازی خود نوعی الگوریتم فشرده‌سازی محسوب می‌شوند. 

در این مستند سعی شده است تا مقدمه‌ای برای ورود به دنیای شگفت‌انگیز الگوریتم‌های فشرده‌سازی 
نوشته شود و خوانندگان را با تعاریف بنیادین و کاربردهای اصلی این الگوریتم‌ها آشنا کند، این مستند مقدمتاً دربارهٔ تعاریف اصلی و 
انواع الگوریتم‌های فشرده‌سازی و کاربردهای اصلی آن‌ها در دنیای امروز بحث می‌کند و پس از آن به بررسی جزیی‌تر دو فرمت 
اصلی فشرده‌سازی (
  \lr{mp3} و jpeg
) می‌پردازد. امید است تا مطالب این مستند مورد توجه خوانندگان عزیز قرار گیرد. 

% به عنوان موخره چکیده باید نوشت که این مستند در راستای انجام پروژهٔ مستند‌سازی درس ارائه مطالب علمی و فنی تهیه شده است و 
% برای سهولت کار استاد محترم درس برای تحصیل اطمینان از درستی مستندسازی و همچنین استفاده دانش‌جویان علاقه‌مند، سیر پیشرفت مستند به همراه کدهای
%   	\LaTeX  \space
%   در 
% \textit{  \href{https://github.com/merfanian/DataCompressionDoc}{Github} 
% }  قرار گرفته اند، لازم به ذکر است که این مستند به صورت متن‌باز ارائه شده و 
% استفاده از آن بدون ذکر منبع برای همگان آزاد است. \\
  \vspace{10mm}

  
    \textbf{ کلمات کلیدی: الگوریتم‌های فشرده‌سازی، فشرده‌سازی  تصویر، فشرده‌سازی صوت، \lr{mp3}، JPEG}
\end{abstract}