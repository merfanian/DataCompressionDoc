\chapter{جمع‌بندی}
\noindent
\textbf{
	\textit{
      مروری بر مباحث این مقاله، هدف مقاله و نتیجهٔ پایانی
	}
}
\pagebreak

\section{خلاصه}
در این مقاله سعی شد تا نگاهی گذرا به مسالهٔ الگوریتم‌های فشرده‌سازی در علوم کامپیوتر در هر دو حیطهٔ عملی و نظری انداخته شود.
ابتدا دربارهٔ تعاریف و انواع الگوریتم‌های فشرده‌سازی نوشتیم و سپس به کاربردهای مختلف این الگوریتم‌های در قسمت‌های مختلف علوم کامپیوتر پرداختیم؛ 
در فصل دوم و سوم نیز بر روی دو محصول نهایی این الگوریتم‌ها متمرکز شدیم، دو فرمت برای نمایش تصویر و صوت که 
هر دو در بطن خود از الگوریتم‌های فشرده‌سازی مختلفی استفاده می‌کردند. 
برای درک نحوهٔ کار این دو فرمت نیاز بود تا مقدمه‌ای بر تبدیل کسینوسی گسسته بنویسیم و سپس گریزی به برخی مفاهیم فراعلوم‌کامپیوتری 
بزنیم. این که ساختار چشم انسان قادر به تشخیص چه فرکانس‌هایی نیست و گوش انسان چه صداهایی را نمی‌شنود مهم‌ترین مباحث فراعلوم‌کامپیوتری این 
مقاله بودند. در این میان الگوریتم‌های مختلف فشرده‌سازی با هدررفت داده و بدون هدررفت داده با مثال‌هایی برای درک بهتر
بررسی شدند و در این فصل هم این مقاله به پایان می‌رسد.

\section{نتیجه‌گیری}
به علت واگویی ساده‌تر مباحث فشرده‌سازی این مقاله نتیجه‌گیری علمی خاصی ورای مباحثی که تا به حال در این مورد نوشته شده ندارد اما 
آشنا شدن بیشتر با کاربردهای فشرده‌سازی در دنیای واقع و درک بهتر برخی الگوریتم‌های فشرده‌سازی از فواید این مقاله برای نگارنده بود. می‌توان به عنوان
اصلی‌ترین نتیجهٔ این مقاله نوشت که الگوریتم‌های فشرده‌سازی در دنیای علوم کامپیوتر یکی از جذاب‌ترین و مهم‌ترین زمینه‌های تحقیق
می‌باشند که با گذشت زمان پویایی خود را حفظ کرده‌اند. هم‌اکنون نیز تحقیقات بسیاری برای تسریع عملیات فشرده‌سازی و بازیابی اطلاعات بر روی رشته‌های دی‌ان‌ای 
در حال انجام است، علاوه بر ژنتیک در زمینهٔ امنیت نیز پژو‌هش‌های زیادی شکل گرفته و با گذشت زمان نیز در حال شکل‌گیری‌ست. 

\section{هدف مقاله و حق استفاده}
هدف اصلی این مقاله نوشتن مقدمه‌ای مختصر و ساده برای آشنایی علافه‌مندان به علوم کامپیوتر و ساختار داده با یکی از قسمت‌های مهم
این رشته یعنی الگوریتم‌های فشرده‌سازی بود، در نگارش این مقاله سعی شد تا حتی‌الامکان مباحث به صورت ساده و با مثال عینی بیان شوند و از 
تکلف‌ورزی در بیان مطالب یا ورود جزیی به بنیان‌های عمیق ریاضیاتی مساله جلوگیری شود. 

همچنین سعی شد تا پرکاربردترین فرمت‌های فشرده‌سازی در این مقاله بررسی شوند تا مخاطبان در حین مطالعه درک کلی از کاربرد 
الگوریتم‌های فشرده‌سازی داشته باشند و مباحث مورد بحث کاملا انتزاعی و دور از مباحث دنیای واقع دیده نشوند. 

مؤخراً باید نوشت که این مستند در راستای انجام پروژهٔ مستند‌سازی درس ارائه مطالب علمی و فنی تهیه شده است و 
برای سهولت کار استاد محترم درس برای تحصیل اطمینان از درستی مستندسازی و همچنین استفاده دانش‌جویان علاقه‌مند، سیر پیشرفت مستند به همراه کدهای
  	\LaTeX  \space
  در 
\textit{  \href{https://github.com/merfanian/DataCompressionDoc}{Github} 
}  قرار گرفته اند، لازم به ذکر است که این مستند به صورت متن‌باز ارائه شده و 
استفاده از آن بدون ذکر منبع برای همگان آزاد است. \\