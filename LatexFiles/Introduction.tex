\chapter{مقدمه}
\noindent
\textbf{
	\textit{
		توضیحی اولیه مبنی بر تعریف کلی فشرده‌سازی، انواع الگوریتم‌ها و کاربردها
	}
}
\pagebreak
\section{تعریف}

الگوریتم‌های فشرده‌سازی، الگوریتم‌هایی هستند که می‌توان با استفاده از آن‌ها داده‌ها را طوری رمزنگاری کرد که در تعداد کمتری بیت نسبت به آرایش اولیه
قابل ارائه باشند.

برای مثال می‌دانیم که برای ذخیرهٔ هر بیت اسکی هشت بیت فضا لازم است، می‌توان با استفاده از نگاشتی متشکل از حروف استفاده‌شده در 
یک متن تعداد بیت‌های مورد نیاز برای نشان دادن هر حرف استفاده‌شده در متن را کاهش داد.

\section{انواع الگوریتم‌های فشرده‌سازی}

در یک دسته‌بندی الگوریتم‌های فشرده‌سازی را به دو نوع زیر افراز می‌کنند.
\begin{itemize}
	\item Lossless
	\item Lossy
\end{itemize}

\subsection{الگوریتم‌های Lossless}
در این سری الگوریتم‌ها دادهٔ ورودی بدون هیچ‌گونه هدررفتی از دادهٔ خروجی قابل بازیابی‌ست، این الگوریتم‌ها در مواقعی که 
ثابت ماندن داده در طی فشرده‌سازی الزامی‌ست استفاده می‌شوند، همچنین معمولا برای بازیابی اطلاعات فشرده‌شده نیاز به 
داده‌هایی خارجی است که با کمک آن عمل بازیابی انجام می‌گیرد، از این رو می‌توان از این نوع الگوریتم‌ها در رمزنگاری نیز 
استفاده کرد. کاربرد اصلی این الگوریتم‌ها در فشرده‌سازی متون است که اشتباه شدن حتی یک حرف می‌تواند باعث بدخوانی و 
بدفهمی متن اصلی گردد. الگوریتم‌های مشهور کمپرس متن به شرح زیر اند.
\lr{
\begin{itemize}
	\item RLE
	\item Huffman Encoding
	\item Burrows Wheeler Transform
\end{itemize}
}


\subsection{الگوریتم‌های Lossy}

در این الگوریتم‌ها پس از هر بار فشرده‌سازی مقداری از داده‌ها از دست می‌روند، معیار ارزیابی این الگوریتم‌ها 
مقدار فشرده‌سازی با توجه به میزان هدررفت داده می‌باشد، به علت هدررفت مقداری از داده این الگوریتم‌ها معمولا در 
مواردی که هدررفت اندک داده توسط انسان یا ماشین قابل تشخیص نباشد استفاده می‌شوند،‌ مثلا تکنیک‌های 
ذخیره‌سازی تصاویر و ویدئو‌ها در کامپیوترها مبتنی بر الگوریتم‌های Lossy 
است زیرا چشم انسان قادر به تشخیص عوض شدن تعدادی پیکسل در صفحه پس از بازیابی فایل فشرده‌شده نیست. 


% insert some figures here 

\section{کاربردها}
فشرده‌سازی داده‌ها در دنیای امروز مهندسی کامپیوتر در نقاط مختلفی نقش دارد، در ادامه مختصرا برخی از کاربردهای 
فشرده‌سازی توضیح داده می‌شود.

\begin{itemize}
	\item مدیا
	
	برای انتقال داده‌های صوتی-تصویری یا به عبارت دیگر مدیاهای مختلف نیاز به الگوریتم‌های فشرده‌سازی به شدت احساس
	می‌شود، حجم هر فیلم سینمایی بدون فشرده‌سازی می‌تواند تا صدها گیگابایت برسد و انتشار هر تصویر در فضای اینترنت
	بدون فشرده‌سازی می‌تواند هر بیننده را ساعت‌ها معطل کند. فرمت‌های مختلف فشرده‌سازی عکس مانند
	JPEG, PNG, EPS, ...
	 و همچنین فرمت‌های پخش ویدئو‌ مانند 
	 MP4, MKV, ...
	 برای اکثر افراد فرمت‌های شناخته‌شده‌ای هستند که در بطن خود از الگوریتم‌های مختلف فشرده‌سازی استفاده می‌کنند.
	\item پردازش سیگنال\\
	
	\item ژنتیک\\
	
	\item امنیت\\
\end{itemize}